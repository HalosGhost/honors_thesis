\documentclass[letterpaper,12pt]{article}

\usepackage{xunicode}
\usepackage{fontspec}
\usepackage[no-sscript,no-logos]{xltxtra}
\usepackage{fancyhdr}
\usepackage{setspace}
\usepackage{tabu}
\usepackage{nameref}
\usepackage[autostyle]{csquotes}
\usepackage[margin=1in]{geometry}
\usepackage[strict,notes,backend=biber]{biblatex-chicago}
\usepackage[multiple,marginal,bottom]{footmisc}

\addbibresource{citations.bib}
\setromanfont[Mapping=tex-text]{Linux Libertine O}
\setlength{\headheight}{15.2pt}

\pagestyle{fancy}
\fancyhf{}
\renewcommand{\headrulewidth}{0pt}
\fancyhead[L]{Framing Political Interaction as Conversation}
\fancyhead[R]{Sam Stuewe}
\fancyfoot[C]{–\ \thepage\ –}

\setstretch{2}
\begin{document}
\title{``I'm not Touching You'':\\ Framing Political Interaction as Conversation}
\author{Sam Stuewe\\ Professor David Blaney 
   \\ Professor Andrew Latham}
%\maketitle
%\thispagestyle{empty}
%\newpage
%\tableofcontents
%\thispagestyle{empty}
%\newpage
\setcounter{page}{1}
\section{Introduction}
For years, political theorists have attempted to use metaphors to simply explain political interaction. 
Perhaps the simplest metaphor used for this purpose hails from the Realist tradition: all actors involved in an interaction are solid billiards balls whose internal setups are of no concern to the analysts and whose interactions with other actors are reduced to hitting one another which may be helpful or hindering.\footcite{mearsheimer01} 
However, like many of the metaphors that have followed, this framework fails to grasp the complexity of the situation at hand. 
Some would argue that the major benefit from these models is that, despite the loss of some detail, they make a complex situation easy to understand. 
Unfortunately, as the saying goes, ``the devil is in the details,'' and dismissing complexity in favor of simplicity only leads to the creation of a model without application. 
More ``leftist'' political theories, such as social constructivism, have offered less overly-simplified metaphors which allow for much more complex understandings, but often lead to such high-level abstraction that it becomes difficult for lay persons (or even modern politicians) to apply them in a given interaction.

In recent years, political interaction has fallen into categories that many of these models have difficulty understanding. 
Most obvious in this category (widely, and problematically, referred to as ``protest politics'') took the shape of the Occupy movements in concert with the ``Arab Spring.'' 
Realism largely disregards these movements as being outside the view of important actors despite their mass impact; social constructivism on the other hand is very capable of discussing the structures which have led to these protests and the systems in which they operate ignoring that the founding principle of these protests was to advocate for an escape and reformation of the structures themselves not of their subordinate policies. 
Thus, I intend to propose a new framework, one that has been in use for a long while but has not yet been formalized. 
Its goal is to provide a level of depth and complexity such that a wide variety of interactions may be fruitfully described and analyzed while remaining simple and accessible enough that people who do not identify as political theorists may still find it useful. 
To be clear, unlike \textit{realpolitik}, this is a framework meant to facilitate understanding, not a model meant to allow prediction. 
Through this framework, I will argue that the Occupy protests were not a fundamentally new type of political interaction, but rather a new manifestation of an older concept. 
Further, I will argue that Occupy is a scalable phenomenon and, therefore, may be applicable to international contexts. 
To make this argument, I must first make my foundational claims clear. 
So I will begin by discussing the origins of this framework and how it has evolved. 
Then, having established its background, I will elaborate on the framework itself and make clear its components. 
Using the elaboration of its components, I will seek to create a functional typology of political interactions. 
And finally, using this framework and typology, I will analyze Occupy on several cases of scale to establish its scalability.

\section{Origins and Evolution}
\label{sec:origins}
[Introduction to Lit Review]
\subsection{Arenas, space, geography, etc.}
The notion of an arena in which political interactions take place is not a new concept and the exact meaning of the term has not always well-defined. 
For some political scientists, it defines, at its most basic level, a sense of scope---e.g., the ``international arena.'' 
However, for Henry Mintzberg, for example, an arena is less a definition of scope as much as it is a description of longevity. 
In his text ``The Organization as Political Arena,"\footcite{mintzberg85} Mintzberg refers to arenas as being particular situations of political interaction characterized by a form of organized conflict. 
He posits a typology of political arenas for which the defining characteristic which differentiates the cases are how the organization of the arena is constituted (primarily in terms of the intensity of the conflict, how well contained the conflict remains and for how long the conflict lasts).\footcite[141]{mintzberg85} 
Mintzberg's typology is founded upon the notion that all organizations are oriented around conflict and that because of the trends of conflict in politics, there are only four basic types of arena. 
The most useful component of this model is its simplicity. 
As with other seemingly predictive models, given various input factors (of which there are not many possibilities), probable outcomes may be predicted with confidence. 
But the notion of a political arena as being only valuable with reference to conflict explicitly dismisses cooperative political interaction;\footcite[152]{mintzberg85} and, even more distressing, his model assumes a vaccuum-sealed environment. 
That is, there is little to no context for which the arena was shaped before the conflict becomes apparent. 
No political interaction is without context and without giving some analysis focused on the context which gave rise to, or influenced, the interactions themselves, a great deal of useful information (which will assuredly affect outcomes) is lost.

\subsection{Participants}
While the location and context of any political interaction holds great importance, those who participate in the interaction itself are, perhaps, the primary subjects of interest. 
In the intro to his work, \emph{A Semisovereign People}, E. E. Schattschneider frames political interaction as conflict and confrontation as it relates to participants.\footcite{schattschneider75} 
In particular, his model focuses on an argument taking place in a hotel lobby.\footcite[1--5]{schattschneider75} 
In this argument, as in Schattschneider's framework as a whole, the focus should be placed on the ``audience,'' not the ``speakers.'' 
He pays special attention to the size of the audience as this is what, he claims will play the decisive role in determining the outcome of the conflict: 
\blockquote{\setstretch{1}The logical consequence of the foregoing analysis of conflict is that the balance of forces in any conflict is not a fixed equation until \emph{everyone} is involved. If one tenth of 1 percent of the public is involved in conflict, the latent force of the audience is 999 times as great as the active force, and the outcome of the conflict depends overwhelmingly on what the 99.9 percent do. Characteristically, the potentially involved are more numerous than those actually involved. This analysis has a bearing on the relations between the ``interested'' and the ``uninterested'' segments of the community and sheds light on interest theories of politics. It is hazardous to assume that the spectators are uninterested because a free society maximizes the contagion of conflict; it invites intervention and gives a high priority to the participation of the public in conflict.\footcite[5]{schattschneider75}}\setstretch{2}
Here, Schattschneider offers a great deal of helpful concepts. 
For example, he offers us the concept of interest as it relates to participants (though he makes this distinction as only being important with reference to audience members and not vocal participants). 
However, perhaps most important, he makes obvious that the dynamics of an interaction are ever-changing. 
Unfortunately, Schattschneider's model focuses almost entirely on the participants (and, even then, very heavily on the audience and not the vocal participants). 
Furthermore, not all political interaction can be accurately characterized as conflict. 
Therefore, his model is simply not applicable to a wide variety of cases, and despite the utility of many of his concepts, his model overall is too focused on only one aspect of any given case.

Less influenced by game theory, Susan Bickford still assumes a level of conflict evident in political interaction but focuses on a different part of participants and their relation to the conversation as a whole.
In particular, her text \emph{The Dissonance of Democracy}\footcite{bickford96} focuses on listening and how it affects conversations disproportionately with how much it has been addressed in academic literature. 
Referencing Plato's \emph{Republic}, Bickford argues that communicative interaction is founded on the understanding that those making arguments will have an audience willing to listen.\footcite[3]{bickford96}
This realization is extremely useful as it offers a basic cast of participants and their roles in a conversation.
Fundamentally speaking, arguments rely on having multiple participants who play various roles (moving between orator and interpreter).
Beyond this realization, it highlights an extremely powerful action for audience members to take: not listening.\footcite[3]{bickford96}
Bickford makes note that pre-cursory agreements determine rules of interaction such that various means of decision-making become normalized.\footcite[3--5]{bickford96}\footnote{Some theorists have referred to these agreements as ``constitutions.''}
These agreements are often made to allow for preferring majority rule or another consensus-based mechanism in the place of the use of force.
Bickford's argument is particularly useful because it does not empower ``listening'' to the disenfranchisement of ''speech,'' but rather equalizes the playing field for both claiming that politics itself is ``about the dynamic between the two.''\footcite[4]{bickford96}
Furthermore, it does not assume conflict is the only character of interaction.
She asserts a complex understanding of relationships (through connections and divisions) which influence actions through affects on the dynamics of any given conversation.\footcite[9--11]{bickford96}

[Bickford's last chapter and gap in her literature for critique]

\subsection{Content}
With his focus set on the final aspect of political interaction, Jeffrey Banks's \emph{Signlaing Games in Political Science} concentrates on information transition, or ``signaling.''\footcite{banks91} 
Banks's analysis focuses on signaling games (particularly incomplete information signal games\footnote{He argues that models which allow for complete information have very little predictive and intellectual merit due to serious limitations on the applicability of the model in real situations}) and how political interaction can be cast in the light of these games. 
Despite his focus on game theory (which emphasizes predictiveness), Banks still offers a prototypical model of politics as conversation. 
In his case analysis, he cast the model's ``players'' as speakers and the signals sent which may influence other players' actions as ``speeches.''\footcite[37]{banks91} 
Banks's model greatly limits the scope of its application through its implicit definitions. 
For example, the only signals Banks concerns himself with are those which influence others by limiting the receiver's set of choices through a mechanism he refers to as ``agenda control.''\footcite[3]{banks91} 
Not only does this cut out all cases of interaction in which the effects are undetermined, but it undermines the understanding of interactions as being more than just one way. 
For Banks, there are two well-defined players, the ``sender'' and ``receiver,'' and there is little acknowledgment of any response.\footcite[4]{banks91} 
That is, where Banks imagines, at most, a one-directional message being sent, a more realistic metaphor would describe messages being sent back and forth between multiple participants which often mutually influence one another. 
Furthermore, Banks's notion of influence is a very traditional definition of power (where an actor limits another actor's ability to do something or expressly forces them to do something they otherwise would not) for the sake of the model's application for prediction to the detriment of its complexity. 
However, despite all the limitations Banks places on his model, it remains surprisingly versatile. 
In his case analysis, he applies his model to literal cases (candidates making speeches to audiences) but makes it generalizable through a metaphor by which analysts may understand much higher-level games through the same terms. 
This is what makes Banks's model so appealing; despite all its limitations, it is scalable.

\section{Conversation as a Metaphor}
To attempt to understand any given conversation, there are three basic components of which an observer must take note: the location, the participants and the content. 
Without being able to discuss all three of these components, some vital portion of the situation being analyzed is lost.
Each author discussed before has given a great deal of valuable concepts which, when developed further and synthesized, offer a far more realistic and revealing picture of political interaction.
So, to begin this framework, I will draw the concepts of the authors mentioned above to construct a unified framework for political interaction as conversation.

Following the order of aspects covered in ``\nameref{sec:origins},'' I begin with location.
Imagine a room. In this room, there are an indeterminant number of walls, doors and windows.
Both the doors and windows vary between open and closed states; and on occasion, some are removed entirely or newly installed.
The room itself was built a long time ago (though there was never an exact moment when construction was started).\footnote{Though it is outside the scope of this work to establish, the concept of a well-defined line between a ``state of nature'' and modern society has been soundly questioned. To reflect this, assume that this room has been evolving through construction and maintenance without a clearly defined beginning.}
As a result of previous alterations, this room's acoustics and floor are uneven which amplify some voices while effectively muting others. 
Finally, despite this room's having changed (significantly) over time, alterations made to the room take a great deal of time, and often happen as individual projects to change very small characteristics.


%\section{A Typology of Conversations}
%\section{Case Analysis}
\newpage
\printbibliography
\end{document}
